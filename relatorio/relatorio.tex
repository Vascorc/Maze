\documentclass[12pt,a4paper]{article}
\usepackage[utf8]{inputenc}
\usepackage[portuguese]{babel}
\usepackage{graphicx}
\usepackage{geometry}
\usepackage{hyperref}
\usepackage{listings}
\usepackage{color}
\usepackage{titlesec}

\geometry{
 a4paper,
 total={170mm,257mm},
 left=25mm,
 top=25mm,
 right=25mm,
 bottom=25mm,
}

\begin{document}

\begin{titlepage}
    \centering
    \vspace*{1cm}
    
    % Se tiver o logo da universidade, descomente a linha abaixo e coloque o ficheiro na pasta
    \includegraphics[width=0.4\textwidth]{logo_universidade.png}
    
    \vspace{1.5cm}
    
    \Huge
    \textbf{Computação Gráfica}
    
    \vspace{0.5cm}
    \LARGE
    Relatório do Projeto Final: Labirinto 3D
    
    \vspace{2.0cm}
    
    \textbf{Autores:}
    \vspace{0.5cm}
    \Large

    Alexandre Santos n52011 \\
    Vasco Colaco n52290 \\
    
    \vfill
    
    \Large
    \textbf{Unidade Curricular:} Computação Gráfica \\
    \textbf{Curso:} Licenciatura em Engenharia Informática
    
    \vspace{1.0cm}
    
    \Large
    \today
    
\end{titlepage}

\tableofcontents
\newpage

\section{Introdução}
Este relatório descreve o desenvolvimento do projeto final da unidade curricular de Computação Gráfica. O objetivo principal do trabalho foi a criação de uma aplicação gráfica interativa em 3D, utilizando a API OpenGL, que aplicasse os conhecimentos adquiridos ao longo do semestre.

\subsection{O Jogo: Labirinto 3D}
A aplicação desenvolvida consiste num jogo de exploração em primeira pessoa. O jogador é colocado num labirinto complexo e escuro, equipado apenas com uma lanterna. O objetivo é navegar pelos corredores, evitando obstáculos e encontrando o caminho correto até ao portão de saída.

\subsection{Motivação}
A escolha deste tema deveu-se ao interesse em compreender os fundamentos dos motores de jogo modernos (Game Engines). Implementar um sistema de câmaras, deteção de colisões e iluminação dinâmica "do zero" (utilizando apenas OpenGL e bibliotecas auxiliares básicas) permite uma compreensão profunda da pipeline gráfica e da matemática vetorial envolvida na renderização 3D.

\section{Tecnologias Utilizadas}
Para o desenvolvimento deste projeto, foram selecionadas tecnologias padrão na indústria e na comunidade de aprendizagem de Computação Gráfica:

\begin{itemize}
    \item \textbf{OpenGL (Core Profile 3.3)}: API gráfica principal utilizada para a renderização de primitivas, gestão de buffers e shaders.
    \item \textbf{C++}: Linguagem de programação utilizada, escolhida pela sua performance e controlo sobre a gestão de memória.
    \item \textbf{GLFW}: Biblioteca utilizada para a criação da janela, contexto de renderização e gestão de input (teclado e rato).
    \item \textbf{GLAD}: Biblioteca para carregar os ponteiros das funções OpenGL em tempo de execução.
    \item \textbf{GLM (OpenGL Mathematics)}: Biblioteca de matemática (header-only) compatível com GLSL, utilizada para operações com vetores e matrizes (transformações, projeções, câmara).
    \item \textbf{stb\_image}: Biblioteca para carregamento de texturas (PNG, JPG).
    \item \textbf{TinyOBJLoader}: Biblioteca para importação de modelos 3D no formato Wavefront (.obj).
\end{itemize}

\section{Desenvolvimento e Implementação}

\subsection{Gestão do Projeto e Estrutura}
O projeto foi estruturado de forma orientada a objetos para garantir a modularidade e a manutenibilidade do código. As principais funcionalidades foram encapsuladas em classes específicas.

\subsection{Descrição das Classes e Módulos}

\subsubsection{Main (main.cpp)}
O ficheiro principal contém o ciclo de jogo (\textit{game loop}). É responsável por:
\begin{itemize}
    \item Inicializar o GLFW e GLAD.
    \item Configurar os callbacks de input.
    \item Gerir o cálculo do \texttt{deltaTime} para garantir movimentos independentes da taxa de frames.
    \item Executar o loop de renderização e processamento logico.
\end{itemize}

\subsubsection{Câmara (Camera.h)}
Implementa uma câmara FPS (\textit{First-Person Shooter}). Utiliza ângulos de Euler (Yaw e Pitch) para calcular o vetor diretor da câmara com base no movimento do rato. Processa também o input do teclado para mover a posição da câmara no mundo (W, A, S, D).

\subsubsection{Maze (Maze.h)}
Esta classe é responsável pela gestão do mundo do jogo.
\begin{itemize}
    \item \textbf{Carregamento do Nível}: Lê o ficheiro .obj do labirinto e separa os dados em vértices, normais e coordenadas de textura.
    \item \textbf{Classificação de Superfícies}: Analisa a normal de cada triângulo para distinguir entre "chão" (normais a apontar para cima) e "paredes".
    \item \textbf{Sistema de Colisões (Grid Espacial)}: Para otimizar a deteção de colisões, o mundo 3D é dividido numa grelha 2D (XZ). Cada celula da grelha armazena referências para os triângulos que a intersetam. Quando o jogador se move, o sistema verifica colisões apenas contra os triângulos nas células vizinhas, drasticamente reduzindo o custo computacional.
\end{itemize}

\subsubsection{Gestor de Shaders (Shader\_m.h e Ficheiros GLSL)}
Foi criada uma classe para abstrair a compilação e linkagem dos shaders. O projeto utiliza vários shaders:
\begin{itemize}
    \item \textbf{Basic Lighting}: Implementa o modelo de iluminação (Phong/Blinn-Phong), texturização e a lógica da lanterna (Spotlight).
    \item \textbf{Skybox}: Renderiza o cubo envolvente (céu).
    \item \textbf{Overlay}: Renderiza imagens 2D sobre a cena 3D (GUI).
\end{itemize}

\subsection{Documentação (Doxygen)}
O código fonte foi documentado utilizando comentários compatíveis com Doxygen. Foi gerada uma documentação HTML que descreve a hierarquia de classes, os métodos e as interdependências, facilitando a consulta técnica do projeto.

\section{Conclusões}
O desenvolvimento deste projeto permitiu consolidar os conhecimentos teóricos de Computação Gráfica.

\textbf{Resultados Alcançados:}
\begin{itemize}
    \item Foi criado um ambiente 3D navegável com performance estável.
    \item O sistema de carregamento de mapas é flexível, permitindo trocar o labirinto apenas alterando o ficheiro .obj.
    \item A iluminação e as texturas conferem ao jogo a atmosfera desejada.
    \item O sistema de colisões, embora simples, é robusto o suficiente para a jogabilidade proposta.
\end{itemize}

\textbf{Trabalho Futuro:}
Como melhorias futuras, planeia-se:
\begin{itemize}
    \item Refinar o sistema de sombras (Shadow Mapping) para suportar sombras dinâmicas projetadas pela lanterna.
    \item Implementar um menu inicial interativo.
    \item Adicionar efeitos sonoros para passos e ambiente.
\end{itemize}

\section{Bibliografia}
\renewcommand\refname{} % Remove o título automático se necessário
\begin{thebibliography}{9}

\bibitem{learnopengl}
Joey de Vries,
\textit{LearnOpenGL},
\url{https://learnopengl.com/}.

\bibitem{opengl-docs}
Khronos Group,
\textit{OpenGL 3.3 Reference Pages},
\url{https://www.khronos.org/registry/OpenGL-Refpages/gl4/}.

\bibitem{glm}
G-Truc Creation,
\textit{OpenGL Mathematics (GLM)},
\url{https://glm.g-truc.net/0.9.9/index.html}.

\bibitem{tinyobj}
Syoyo Fujita,
\textit{TinyObjLoader},
\url{https://github.com/tinyobjloader/tinyobjloader}.

\end{thebibliography}

\end{document}
